\documentclass[11pt,a4paper]{report}

\evensidemargin=0cm
\oddsidemargin=0cm
\topmargin=-1cm
\textheight=23.5cm
\leftmargin=0cm
\textwidth=18cm
\sloppy
\flushbottom
\parindent 1em
\hoffset -0.5in
\oddsidemargin  0pt
\evensidemargin 0pt
\marginparsep 10pt

\usepackage[utf8]{inputenc}
\usepackage[frenchb]{babel}
\usepackage[T1]{fontenc}

\usepackage{listings}
\usepackage{color}
\definecolor{lightgray}{rgb}{.9,.9,.9}
\definecolor{darkgray}{rgb}{.4,.4,.4}
\definecolor{purple}{rgb}{0.65, 0.12, 0.82}

\lstnewenvironment{OCaml}
                  {\lstset{
                      language=[Objective]Caml,
                      breaklines=true,
                      commentstyle=\color{purple},
                      stringstyle=\color{red},
                      identifierstyle=\ttfamily,
                      keywordstyle=\color{blue},
                      basicstyle=\footnotesize
                    }  
                  }
                  {}

\title{Rapport de projet\\Débogueur Visuel pour OCaml}
\author{Mathieu Chailloux\\Vincent Botbol}
\date\today

\begin{document}
\maketitle

\chapter{Introduction}
% motivations, ...

\chapter{Ocamldebug}

\section{Présentation}
\section{\'Evénements de débogage}
\section{Communication avec la machine virtuelle}
\section{Limitations et inconvénients} % transition


\chapter{OCabug}

%intro : ce qu'on a fait
%présentation des différentes extensions, installation

\section{Présentation de l'interface}
%transition toplevel -> ocamldebug
\section{Placement par rapport à Ocamldebug}
%organisation du code - couche d'ocamldebug
%nécessite le compilo + lablgtk
\section{Implémentation des extensions}
\section{Améliorations possibles}


\chapter{Annexes}

\section{Exemples}

\end{document}
